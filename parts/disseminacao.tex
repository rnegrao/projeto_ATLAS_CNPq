\chapter{Importância científica do projeto}

\thispagestyle{plain}
Uma vez que este projeto está relacionado com a detecção de radiação ionizante, essa proposta possui um grande potencial de inovação científico-tecnológica, principalmente no que tange à sensores tolerantes à radiação. Além disso, o pesquisador responsável por essa proposta possui experiência no desenvolvimento de instrumentação para física de altas energias, adquirida através do trabalho realizado no projeto de {\it upgrade} do TPC do experimento ALICE no CERN \cite{tpcJINST,tpcNIM,discharge_paper,GSI_REPO,THGEM}, e somado à infraestrutura presente no laboratório do HEPIC do Instituto de Física da Universidade de São Paulo para o desenvolvimento de instrumentação nuclear, permitirá a transferência total de tecnologia relacionada com sensores semicondutores de alta performance para o grupo, bem como ageração de novas tecnologias e aplicações, permitindo consolidar a linha de pesquisa em sensores semicondutores no Departamento de Física Nuclear do Instituto de Física da USP. Esse projeto irá corroborar com programas experimentais atualmente em andamento no grupo relacionado com a espectroscopia e reconstrução de imagens com raios-X \cite{THGEM,NIM,xray}, ampliando a capacidade tecnológica do HEPIC com respeito à detecção de radiação ionizante. 
\thispagestyle{plain}

A vista disso, no âmbito nacional, com as ferramentas criadas neste projeto será possível colaborar com os programas de instrumentação em diversos programas de pesquisa presentes no Brasil tais como o Laboratório Nacional de Luz Síncrotron (LNLS), no que diz respeito ao desenvolvimento de sistemas de detecção. De acordo com os estudos levantados pelos pesquisadores do Sirius e descritas em seu projeto \cite{sirius}, {\it 'existe hoje no Brasil uma oportunidade excepcional para o desenvolvimento de expertise na área de detectores híbridos, visando atender às exigências das linhas de luz do Sirius. Futuramente, essa experiência poderá resultar em desenvolvimentos para as áreas médica, industrial e educacional'}. Isso está alinhado com os objetivos desta proposta.

\thispagestyle{plain}
No final do projeto, o {\it know how} adquirido na linha de pesquisa com sensores semicondutores estará consolidado e pronto para dar continuidade ao trabalho independente deste grupo em novas aplicações bem como na geração de novas tecnologias.
\renewcommand{\cleardoublepage}{}
\renewcommand{\clearpage}{}

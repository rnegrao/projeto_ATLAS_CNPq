% Conteúdo do capitulo 
% 1- Desenvolver e caracterizar sensores LGAD para o upgrade do ATLAS

\chapter{Objetivos do projeto}

% 1 - Descrição do desafio experimental para o ATLAS
O objetivo científico deste projeto é participar do desenvolvimento de um sistema de detecção para a região de pseudo-rapidez frontal que seja capaz de melhorar a precisão da medida de luminosidade do feixe e a reconstrução de partículas no experimento ATLAS, para operar durante a fase de alta luminosidade do LHC ({\it High Luminosity LHC} (HL-LHC)) \cite{HL_LHC,tdr}. 

Para superar esse desafio científico, este projeto concentrar-se-á na pesquisa e desenvolvimento de um sistema de detecção frontal ATLAS-HGTD, cuja a técnica experimental é baseada na medida do tempo de vôo das partículas, medindo intervalos de tempo com resolução da ordem de 20-30ps \cite{tdr}, tornando dessa forma possível associar as partículas ao seu vértice de produção para colisões próton-próton no HL-LHC. Para realizar a construção do HGTD, os sensores do tipo LGAD serão adotados como base tecnológica, tendo em vista que eles apresentam excelentes características com respeito ao compromisso entre ganho e resolução temporal. 
\thispagestyle{plain}

O Detector \textit{High Granularity Timing Detector} (HGTD) é um projeto de atualização de fase II para o detector ATLAS \cite{tdr}. Desse modo, é essencial validar os principais aspectos do detector no CERN em 2021 e 2022 no programa de demonstração do detector. A primeira parte importante do projeto está centrada na validação do desempenho térmico do resfriamento de CO$_{2}$ em uma única linha de leitura usando aquecedores de silício para emular a dissipação de energia dos módulos. Essa validação visa garantir a estabilidade térmica dos módulos HGTD com a placa de resfriamento. Os testes de estabilidade térmica serão realizados no CERN até Outubro de 2021. O pesquisador de pós-doutorado terá papel de liderança nas medições de dissipação de energia e na análise do desempenho térmico dos aquecedores da placa de resfriamento.
\thispagestyle{plain}

Em Outubro de 2021, o foco do projeto do demonstrador mudará para os primeiros módulos completos do HGTD (sensor de silício LGAD + ASIC de interface). Estes serão montados e levados ao CERN, onde serão instalados na placa de resfriamento. O pesquisador participará da montagem dos módulos e PCBs flexíveis que serão montados no CERN. Este procedimento de montagem é uma etapa muito importante para validação do detector, e irá definir o caminho completo de leitura dos sinais dos módulos HGTD para as placas eletrônicas de aquisição periféricas. A cadeia de leitura completa será exercitada e estudada para garantir que uma resolução de tempo de 30ps seja alcançável para o detector no início de 2022.

Além dos módulos para o demonstrador, módulos completos serão testados em campanhas periódicas de teste com feixe na instalação \textit{Super Proton Synchrotron} (SPS) no CERN, quando a instalação reiniciar a execução de prótons em meados de 2021. Durante essas campanhas ao longo de 2022, muitos módulos HGTD serão estudados usando píons de alta energia para entender e validar seu desempenho antes e após a irradiação. O pesquisador participará e concentrará parte de seu tempo no estudo de módulos durante essas campanhas de testes com feixe no CERN. Da mesma forma, o pesquisador também participará da caracterização dos sensores LGAD no laboratório do CERN, produzidos para a construção do detector HGTD. 


%Inicialmente o projeto terá como foco o desenvolvimento da metodologia experimental necessária para a caracterização dos sensores LGAD. Isso será feito através da implantação de técnicas com o objetivo de caracterizar os LGAD em termos de sua corrente de fuga, ganho, uniformidade e resolução temporal, tornando possível diagnosticar com precisão a qualidade dos sensores LGAD.

%Além da caracterização dos sensores LGAD, outro objetivo deste projeto será o de integrá-lo aos diversos serviços e sistemas de modo a construir um protótipo do detector HGTD (\textit{demonstrator}). Com a construção deste protótipo, será possível verificar aspectos importantes do detector tais como a uniformidade térmica ao longo do detector, a eficiência na distribuição de alta e baixa tensão para os sensores LGAD e para a eletrônica de aquisição de dados, a integração dos sensores à eletrônica de aquisição de dados, e a taxa de leitura dos dados.

%A construção do \textit{demonstrator} será fundamental para o desenvolvimento e certificação dos procedimentos e métodos que serão empregados na montagem do detector, na validação dos componentes e materiais utilizados, bem como no estabelecimento das medidas de controle de qualidade para a produção dos módulos do HGTD.

% Conteúdo do capitulo 
% 1- Desenvolver e caracterizar sensores LGAD para o upgrade do ATLAS

\chapter{Objetivos do projeto}

% 1 - Descrição do desafio experimental para o ATLAS
O objetivo científico deste projeto é participar do desenvolvimento de um detector semicondutor para a região de pseudo-rapidez frontal que seja capaz de melhorar a precisão da medida de luminosidade do feixe e a reconstrução de partículas no experimento ATLAS, para operar durante a fase de alta luminosidade do LHC ({\it High Luminosity LHC} (HL-LHC)) \cite{HL_LHC,tdr}. 

Para superar esse desafio científico, este projeto irá trabalhar na pesquisa e desenvolvimento de um sistema de detecção frontal ATLAS-HGTD, cuja a técnica experimental é baseada na medida do tempo de vôo das partículas, medindo intervalos de tempo com resolução da ordem de 20-30ps \cite{tdr}, tornando dessa forma possível associar as partículas ao seu vértice de produção para colisões próton-próton no HL-LHC. Para realizar a construção do HGTD, os sensores do tipo LGAD serão adotados como base tecnológica, tendo em vista que eles apresentam excelentes características com respeito ao compromisso entre ganho e resolução temporal. 
\thispagestyle{plain}

Inicialmente o projeto terá como foco o desenvolvimento da metodologia experimental necessária para a caracterização dos sensores LGAD. Isso será feito através da implantação de técnicas com o objetivo de caracterizar os LGAD em termos de sua corrente de fuga, ganho, uniformidade e resolução temporal, tornando possível diagnosticar com precisão a qualidade dos sensores LGAD.

Além da caracterização dos sensores LGAD, outro objetivo deste projeto será o de integrá-lo aos diversos serviços e sistemas de modo a construir um protótipo realístico do detector HGTD (\textit{demonstrator}). Com a construção deste protótipo, será possível verificar aspectos importantes do detector tais como a uniformidade térmica ao longo do detector, a eficiência na distribuição de alta e baixa tensão para os sensores LGAD e para a eletrônica de aquisição de dados, a integração dos sensores à eletrônica de aquisição de dados, e a taxa de leitura dos dados.

A construção do \textit{demonstrator} será fundamental para o desenvolvimento e certificação dos procedimentos e métodos que serão empregados na montagem do detector, na validação dos componentes e materiais utilizados, bem como no estabelecimento das medidas de controle de qualidade para a produção dos módulos do HGTD.


% 1- Desenvolver e caracterizar sensores LGAD para o upgrade do ATLAS
%O objetivo deste projeto é caracterizar e desenvolver sensores semicondutores do tipo LGAD para o upgrade do experimento ATLAS. O pesquisador responsável participará ativamente na qualificação dos sensores e na consolidação de suas especificações. 

%Além disso, o pesquisador participará na construção do protótipo do HGTD. Esse trabalho será fundamental para o desenvolvimento dos métodos e técnicas que serão empregadas na montagem do detector, bem como certificar os materiais empregados.

%2- Estudar experimentalmente os processos físicos relacionados com a amplificação da carga
%Em seguida, com a implantação das diversas metodologias experimentais para a caracterização dos sensores semicondutores será possível estudar em grande detalhe os processos físicos relacionados com a amplificação de carga e a produção de sinal no material semicondutor, visando compreender e melhorar os processos de fabricação baseando-se no aumento do ganho, resolução temporal e estabilidade elétrica dos sensores durante sua operação em ambiente com alta radiação \cite{tdr}. 

%Como resultado espera-se produzir novas gerações de sensores os quais poderão ser utilizados não apenas para a detecção de partículas carregadas, mas como detector de raios-X. Devido às excelentes características dos detectores do tipo LGAD, as quais incluem sua alta eficiência quântica para uma grande faixa de comprimentos de onda e a possibilidade de construir detectores com alta granularidade, aplicações em diversos ramos envolvendo a detecção de raios-X, tais como luz síncrotron, tornam-se muito atrativas e de fácil implantação uma vez estabelecida essa tecnologia. Essa também é uma das propostas do projeto para longo prazo.

%4- Desenvolvimento de um sistema de aquisição
%Além do desenvolvimento dos sensores LGAD, um outro objetivo deste projeto será o de integrá-lo a um sistema de aquisição de dados, o que tornará possível a reconstrução de eventos. Como resultado espera-se desenvolver todas as competência e habilidades presentes nos diversos componentes que compõem o sistema de aquisição, desde os aspectos físicos relacionados com a produção do sinal até o tratamento dos dados e imagens produzidas.

%5- perspectivas de outros trabalhos
%Por fim, o desenvolvimento de algorítimos para a reconstrução dos dados do experimento ATLAS utilizando o HGTD em conjunto com o ITk não serão inclusos de início como objetivos neste projeto, no entanto tendo em vista a importância deste componente para o desenvolvimento do sistema de detecção e aquisição o mesmo poderá, como uma perspectiva futura, ser trabalhado mais adiante no projeto. %dependendo da quantidade de recursos disponíveis para serem realocados para essa atividade.  

% Este projeto tem como objetivo estrategico preparar o terreno para projetos futuros no campo de semicondutores e para a continuacao com projetos tematicos.
\renewcommand{\cleardoublepage}{}
\renewcommand{\clearpage}{}
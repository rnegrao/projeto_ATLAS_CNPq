O Detector \textit{High Granularity Timing Detector} (HGTD) é um projeto de atualização de fase II para o detector ATLAS. É essencial validar os principais aspectos do detector no CERN em 2021 e 2022 no programa de demonstração. A primeira parte importante do projeto está centrada na validação do desempenho térmico do resfriamento de CO$_{2}$ em uma única linha de leitura usando aquecedores de silício para emular a dissipação de energia dos módulos. Essa validação visa garantir a estabilidade térmica dos módulos HGTD com a placa de resfriamento. Os testes de estabilidade térmica serão realizados no CERN até Outubro de 2021. O pesquisador de pós-doutorado terá papel de liderança nas medições de dissipação de energia e na análise do desempenho térmico dos aquecedores da placa de resfriamento.

Em Outubro de 2021, o foco do projeto do demonstrador mudará para os primeiros módulos HGTD de tamanho normal (sensor de silício LGAD + ASIC de interface). Estes serão montados e levados ao CERN, onde serão instalados na placa de resfriamento. O pesquisador participará da montagem dos módulos e PCBs flexíveis que serão montados no CERN. Este procedimento de montagem é uma etapa muito importante para validação do detector final, e irá definir o caminho completo de leitura dos módulos HGTD para as placas eletrônicas periféricas. A cadeia de leitura completa será exercitada e estudada para garantir que uma resolução de tempo de 30ps seja alcançável para o detector no início de 2022.

Além dos módulos para o demonstrador, módulos completos serão testados em campanhas periódicas de teste com feixe na instalação \textit{Super Proton Synchrotron} (SPS) no CERN, quando a instalação reiniciar a execução de prótons em meados de 2021. Durante essas campanhas ao longo de 2022, muitos módulos HGTD serão estudados usando píons de alta energia para entender e validar seu desempenho antes e após a irradiação. O pesquisador participará e concentrará parte de seu tempo no estudo de módulos durante essas campanhas de testes com feixe no CERN. Da mesma forma, o pesquisador também participará da caracterização dos sensores LGAD no laboratório do CERN, produzidos para a construção do detector HGTD. 

%%%%%%%%%%%%%%%%%%%%%%%%%%%%%%%%%%%%%%%%%%%%%%%%%%%%%%%%%%%%%%%%%%%%%

Eu gostaria de contribuir inicialmente através da promoção de um ambiente propício e aberto a discussão de ideias inovadoras que envolvam física, ciências em geral, engenharia e computação que possam resultar, em um segundo momento, na criação de projetos experimentais, com resultados reais, em diversos campos estratégicos, tais como sensores semicondutores, microeletrônica e instrumentação científica avançada, com grande potencial de impactar a sociedade Brasileira.

Eu acredito que o Insper e um ambiente fértil - pelo sua cultura interdisciplinar e multicultural - para a criação de um ambiente de inovação com um potencial ilimitado para a geração de conhecimento sobretudo nas engenharias.


Possuí graduação em Física pela Universidade Estadual de Maringá (2005), mestrado em física nuclear pela Universidade de São Paulo (2009) e doutorado em física nuclear de altas energias pela Universidade de São Paulo (2014). 

Possuí experiência profissional em Física Nuclear Experimental com trabalhos realizados no Brookhaven National Laboratory (BNL), como pesquisador visitante, e no European Organization for Nuclear Research (CERN), como Pos- doutor.